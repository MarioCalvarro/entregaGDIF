\section{Apartado 1}
Supongamos en primer lugar que $\alpha$ es una curva plana parametrizada por
longitud de arco. Recordemos\footnote{Definición 6.9 \cite{sanjurjo}} que la
curva definida por los centros de curvatura de $\alpha$ tiene la siguiente
expresión:
\[
\beta\left( s \right) = \alpha\left( s \right) + \frac{1}{\kappa\left( s
\right)} \mathbf{n}\left( s \right).
\]
Ya que $\alpha$ es una curva plana, sabemos que su torsión es nula, es decir,
$\mathbf{n}'_{\alpha}\left( s \right) = - \kappa\left( s
\right)\alpha'\left( s \right)$. Si derivamos ahora la definición de $\beta$
tenemos que:
\begin{align*}
    \beta'\left( s \right) &= \alpha'\left( s \right) +
\frac{\mathbf{n}'_{\alpha}\left( s \right) \cdot \kappa\left( s \right) -
\kappa'\left( s \right) \cdot \mathbf{n}_{\alpha}\left( s
\right)}{\kappa^2\left( s \right)} \\
&= \alpha'\left( s \right) -
\frac{\kappa'\left( s \right) \cdot \mathbf{n}_{\alpha}\left( s
\right)}{\kappa^2\left( s \right)} + \frac{1}{\kappa\left( s \right)} \left(
-\kappa\left( s \right) \alpha'\left( s \right) \right) = \boxed{-\frac{\kappa'\left( s
\right)}{\kappa^2\left( s \right)} \mathbf{n}_{\alpha}\left( s \right)}
\end{align*}
Con esta última expresión, y sabiendo que $\frac{-\kappa'\left( s
\right)}{\kappa^2\left( s \right)}$ es un escalar y que
$\mathbf{n}_{\alpha}\left( s \right)$ es perpendicular a $\alpha'$, tenemos que
$\alpha'\left( s \right)$ y $\beta'\left( s \right)$ son ortogonales. Ahora,
despejando $\alpha$ en la definición de $\beta$, vemos que
\[
\alpha\left( s \right) = \beta\left( s \right) - \frac{1}{\kappa\left( s
\right)} \mathbf{n}\left( s \right) = \beta\left( s \right) + \frac{\kappa\left(
s\right)}{\kappa'\left( s \right)} \beta'\left( s \right)
\]
y, por tanto, $\alpha\left( s \right)$ se encuentra en la recta tangente a
$\beta\left( s \right)$. Esto\footnote{Definición 7.1\cite{sanjurjo}} nos indica
que $\alpha$ es la \textit{involuta} de $\beta$ con lo que queda demostrada esta
primera implicación.

Antes de ver la implicación recíproca, veamos la unicidad de esta evoluta plana.
Sea pues $\gamma$ una evoluta cualquiera de $\alpha$. Tendremos entonces que
$\alpha\left( s \right) = \gamma\left( s \right) + f\left( s
\right)\gamma'\left( s \right)$ siendo $\alpha'\left( s \right)$ y
$\gamma'\left( s \right)$ ortogonales. Como ambas curvas son planas el hecho de
que sus tangentes sean perpendiculares en todo momento, nos indica que
pertenecen al mismo plano. Además, tenemos que $\gamma'\left( s \right)$ es
proporcional a $\mathbf{n}_{\alpha}\left( s \right)$. En definitiva,
\[
\alpha\left( s \right) = \gamma\left( s \right) + g\left( s
\right)\mathbf{n}_{\alpha} \left( s \right),
\]
luego $\gamma\left( s \right) = \alpha\left( s \right) - g\left( s
\right)\mathbf{n}_{\alpha}\left( s \right)$ y
\begin{align*}
    \gamma'\left( s \right) &= \alpha'\left( s \right) - g'\left( s \right)
    \mathbf{n}_{\alpha}\left( s \right) - g\left( s
    \right)\mathbf{n}'_{\alpha}\left( s \right)\\
    &= \mathbf{t}_{\alpha}\left( s \right) - g'\left( s
    \right)\mathbf{n}_{\alpha}\left( s \right) + g\left( s \right)
    \kappa_{\alpha}\left( s \right) \mathbf{t}_{\alpha}\left( s \right)
\end{align*}
Ahora, haciendo el producto escalar por $\mathbf{t}_{\alpha}\left( s \right)$
sobre esta expresión, tenemos que $0 = 1 + g\left( s
\right)\kappa_{\alpha}\left( s \right)$, debido a que $\mathbf{t}_{\alpha}\left(
s\right)$ es perpendicular a $\gamma'\left( s \right)$ y a
$\mathbf{n}_{\alpha}\left( s \right)$. Con esto obtenemos que $g\left( s \right)
= -\frac{1}{\kappa_{\alpha}}\left( s \right)$ y sustituyendo en la anterior
expresión de $\gamma\left( s \right)$ tenemos que:
\[
\gamma\left( s \right) = \alpha\left( s \right) + \frac{1}{\kappa_{\alpha}\left(
s \right)} \mathbf{n}_{\alpha}\left( s \right)
\]
Es decir, la curva formada por los centros de curvatura de $\alpha$.

Veamos ahora la implicación inversa. Por tanto, supongamos que $\beta\left( t
\right)$, como curva formada por los centros de curvatura de $\alpha$, es
evoluta de $\alpha$. Al ser evoluta se da\footnote{Proposición
7.2\cite{sanjurjo}} la siguiente relación:
\[
\alpha\left( s \right) = \beta\left( s \right) + \left( \lambda - s \right)
\beta'\left( s \right),\ \lambda \in \mathbb{R}
\]
Si sustituimos el valor de $\alpha\left( s \right)$ dado por esta fórmula en la
igualdad de la curva de centros tenemos lo siguiente:
\begin{align*}
    \beta\left( s \right) &= \alpha\left( s \right) + \frac{1}{\kappa\left( s
    \right)} \mathbf{n}_{\alpha}\left( s \right)\\
    &= \beta\left( s \right) + \left( \lambda - s \right) \beta'\left( s
    \right) + \frac{1}{\kappa\left( s \right)} \mathbf{n}_{\alpha}\left( s \right)
\end{align*}
Es decir, $0 = \left( \lambda - s \right)\beta'\left( s \right) +
\frac{1}{\kappa\left( s \right)} \mathbf{n}_{\alpha}\left( s \right)$. En
definitiva, $\beta'\left( s \right) = \frac{1}{\kappa\left( s \right) \cdot
\left( \lambda - s \right)} \mathbf{n}_{\alpha}\left( s \right)$. Como es para
un punto fijo, la anterior igualdad nos indica que $\beta'$ y
$\mathbf{n}_{\alpha}\left( s \right)$ son proporcionales. %TODO: Seguro?
Si calculamos ahora $\beta'$ en base a la fórmula de los centros de curvatura
tenemos que:
\[
\beta'\left( s \right) = \alpha'\left( s \right) + \frac{1}{\kappa\left( s
\right)} \mathbf{n}'_{\alpha}\left( s \right) - \frac{\kappa'\left( t
\right)}{\kappa^2\left( t \right)} \mathbf{n}_{\alpha}\left( s \right)
\]
Sin embargo, como $\alpha'$ y $\mathbf{n}'$ son perpendiculares ambos a
$\mathbf{n}_{\alpha}$, no es posible, haciendo una suma de ambos, obtener un
vector proporcional a $\mathbf{n}_{\alpha}$, luego, su suma en este caso será
$0$:
\begin{align*}
    0 &= \alpha'\left( s \right) + \frac{1}{\kappa\left( s
\right)}\mathbf{n}'_{\alpha}\left( s \right)\\
      &\Rightarrow -\alpha'\left( s \right) \kappa\left( s \right) =
      \mathbf{n}'_{\alpha}\left( s \right)
\end{align*}
Lo que comparado con la fórmula de Frenet-Serret nos indica que la torsión de
$\alpha$ es $0$ y, por tanto, es plana.
