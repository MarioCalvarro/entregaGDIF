\section{Apartado 3}
Veamos ahora que, con las condiciones del anterior apartado, todas las evolutas
son hélices generalizadas, es decir, que los vectores tangentes de todas ellas
forman un ángulo constante con un vector fijo no nulo. Esta definición es
equivalente a que, para un $\mathbf{v}$ fijo, el producto escalar de
$\mathbf{t}_{\beta}\left( s \right)$, siendo $\beta$ una evoluta de $\alpha$, con $\mathbf{v}$
es constante. O lo que es lo mismo,
\[
    \frac{d}{ds}\langle \mathbf{t}_{\beta}\left( s \right), \mathbf{v} \rangle
    = 0,\ \forall s \in I.
\]
Ahora, como la base del cilindro sobre el que están las evolutas de $\alpha$ es
la única evoluta plana ($\hat{\beta}$), y $\mathbf{b}_{\alpha}$ es constante (al
ser $\alpha$ una curva plana birregular) es lógico considerar
$\mathbf{b}_{\alpha}$ como el vector $\mathbf{v}$ de la anterior ecuación.
Veamos que esta ecuación se cumple para todas las evolutas de $\alpha$.

Empezamos por la evoluta plana $\hat{\beta}$. Simplemente como esta curva es
plana y está en el mismo plano que $\alpha$ (perpendicular a
$\mathbf{b}_{\alpha}$) tenemos directamente el resultado.

Veamos ahora el resultado para el resto de evolutas $\beta$. Derivando el
producto escalar tenemos que
\[
0 = \frac{d}{ds}\langle \mathbf{t}_{\beta}\left( s \right),
\mathbf{b}_{\alpha} \rangle = \langle
\mathbf{t}_{\beta}'\left( s \right), \mathbf{b}_{\alpha} \rangle
+ \langle \mathbf{t}_{\beta}\left( s \right),
\mathbf{b}_{\alpha}' \rangle,
\]
pero como $\mathbf{b}_{\alpha}$ es constante, su derivada se anula y nos queda
que
\[
0 = \langle \mathbf{t}_{\beta}'\left( s \right), \mathbf{b}_{\alpha} \rangle.
\]
Si recordamos ahora las fórmulas de Frenet-Serret
\[
\mathbf{t}_{\beta}'\left( s \right) = \kappa_{\beta}\left( s \right)\mathbf{n}_{\beta}\left( s \right)
\]
y sustituimos en la anterior igualdad
\[
0 = \langle \kappa_{\beta}\left( s \right)\mathbf{n}_{\beta}\left( s \right),
\mathbf{b}_{\alpha} \rangle = \underbrace{\kappa_{\beta}\left( s \right)}_{\neq 0}\langle
\mathbf{n}_{\beta}\left( s \right), \mathbf{b}_{\alpha} \rangle,
\]
solo nos quedará por ver que $\mathbf{n}_{\beta}\left( s \right)$ es
perpendicular a $\mathbf{b}_{\alpha}$. Pero esto será cierto ya que $\beta$ esta
contenido en un cilindro construido en la dirección de $\mathbf{b}_{\alpha}$ y
$\mathbf{n}_{\beta}\left( s \right)$ está en la dirección radial de dicho
cilindro.\footnote{Que la curvatura de $\beta$ fuese $0$ solo se podría dar si
$\mathbf{t}_{\beta}'$ también lo fuese, es decir, si $\beta$ fuese una recta.}
