\section{Enunciado}
Dadas dos curvas $\alpha, \beta: I \rightarrow \mathbb{R}^3$, siendo $\beta$
regular, decimos que $\beta$ es una \textit{evoluta} de $\alpha$ si
$\alpha\left( t \right)$ está sobre la recta afín tangente a $\beta$ en
$\beta\left( t \right)$ y, además, $\alpha'\left( t \right)$ y $\beta'\left( t
\right)$ son ortogonales. Se pide:
\begin{enumerate}
    \item Probar que si $\alpha: I \rightarrow \mathbb{R}^3$ es una curva
        regular con $\kappa, \kappa' \neq 0$ en todo punto (aquí $\kappa$ denota
        la función curvatura de $\alpha$), la curva $\beta$ definida por los
        centros de curvatura de $\alpha$ es una evoluta suya si y solo si
        $\alpha$ es una curva plana. En este caso, pruébese además que $\beta$
        es la única evoluta plana de $\alpha$.

    \item Si $\alpha$ es una curva plana regular con $\kappa, \kappa' \neq 0$ en
        todo punto, probar que todas las evolutas de $\alpha$ tienen su traza
        contenida en un cilindro perpendicular al plano que contiene a $\alpha$
        y cuya base es la única evoluta plana de $\alpha$.

    \item En las condiciones del ejercicio anterior, probar que todas las
        evolutas de $\alpha$ son hélices generalizadas.
\end{enumerate}
