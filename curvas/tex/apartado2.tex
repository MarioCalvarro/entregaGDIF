\section{Apartado 2}
Ya que estamos tratando con cilindros generalizados, estos podrán tener como
base cualquier tipo de curva. Estos cilindros tienen la siguiente
parametrización:
\begin{align*}
    \varphi: \mathbb{R}^2 &\rightarrow \mathbb{R}^3\\
    \left( u, v \right) &\mapsto \gamma\left( u \right) + v \mathbf{v}
\end{align*}
donde $\mathbf{v}$ es un vector que nos indica la dirección en la que se
construye el cilindro y $\gamma$, una curva (que será su base). Con esto, para
ver que una curva esta contenida en un cilindro será simplemente necesario ver
que la proyección de dicha curva a un plano perpendicular al cilindro es igual a
la proyección de la propia curva $\gamma$. En nuestro caso tendremos que, como
$\beta$ es una evoluta de $\alpha$ (parametrizada por la longitud de
arco),\footnote{Teorema 7.5\cite{sanjurjo}}
\[
\beta\left( s \right) = \alpha\left( s \right) + \frac{1}{\kappa\left( s
\right)}\mathbf{n}_{\alpha}\left( s \right) + \frac{1}{\kappa\left( s \right)}
\tan\left( \int_{s_0}^{s} \tau\left( u \right) \mathrm{d}u + c \right)
\mathbf{b}_{\alpha}\left( s \right).
\]
Si llamamos ahora $\hat{\beta}$ a la evoluta plana que, por el anterior
apartado, cumple que
\[
\beta\left( s \right) = \alpha\left( s \right) + \frac{1}{\kappa\left( s
\right)} \mathbf{n}_{\alpha}\left( s \right).
\]
Restando ambas ecuaciones,
\[
\beta\left( s \right) - \hat{\beta}\left( s \right) =
\underbrace{\frac{1}{\kappa\left( s \right)} \tan\left( \int_{s_0}^{s}
\tau\left( u \right) \mathrm{d}u + c \right)}_{= \lambda}
\mathbf{b}_{\alpha}\left( s \right),
\]
nos quedamos con que
\[
\beta\left( s \right) = \hat{\beta}\left( s \right) + \lambda
\mathbf{b}_{\alpha}\left( s \right)
\]
lo que finalmente nos indica que $\beta$ se encuentra en un cilindro con base
$\hat{\beta}$. Como esta última curva es plana y se encuentra en el mismo plano
que $\alpha$, tenemos el resultado.
