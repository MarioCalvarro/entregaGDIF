\section{Apartado 2}
En primer lugar, recordamos que la parametrización que hemos elegido es:
\[
\varphi\left( \lambda, \theta \right) = \left( \cosh \lambda \cos \theta,
\cosh \lambda \sin \theta, \lambda \right).
\]
Siendo la definición de primera forma fundamental:
\begin{align*}
    \mathrm{I}_p: T_pX &\rightarrow \mathbb{R}\\
    \left( v, v' \right) &\mapsto \langle v, v' \rangle
\end{align*}
tiene como matriz respecto de la base $\langle \varphi_{\lambda}\left( p \right),
\varphi_{\theta}\left( p \right) \rangle$
\[
    \begin{pmatrix} E & F\\ F & G \end{pmatrix} = \begin{pmatrix} 
    \varphi_{\lambda} \cdot \varphi_{\lambda} & \varphi_{\lambda} \cdot \varphi_{\theta}\\
    \varphi_{\theta} \cdot \varphi_{\lambda} & \varphi_{\theta} \cdot
    \varphi_{\theta}
    \end{pmatrix}
\]
Por tanto, como:
\begin{gather*}
    \varphi_{\lambda} = \left( \sinh \lambda \cos \theta, \sinh \lambda \sin
    \theta, 1 \right)\\
    \varphi_{\theta} = \left( -\cosh \lambda \sin \theta, \cosh \lambda \cos
    \theta, 0 \right)
\end{gather*}
Tenemos que:
\begin{align*}
    \varphi_{\lambda} \cdot \varphi_{\lambda} &= \sinh^2 \lambda \cos^2 \theta +
    \sinh^2 \lambda \sin^2 \theta + 1 = 1 + \sinh^2 \lambda\\
    \varphi_{\theta} \cdot \varphi_{\theta} &= \cosh^2 \lambda \sin^2 \theta +
    \cosh^2 \lambda \cos^2 \theta = \cosh^2 \lambda\\
    \varphi_{\lambda} \cdot \varphi_{\theta} = \varphi_{\theta} \cdot
    \varphi_{\lambda} &= -\cosh \lambda \sinh \lambda \cos \theta \sin \theta +
    \cosh \lambda \sinh \lambda \cos \theta \sin \theta = 0
\end{align*}
Con lo que al final la matriz queda como:
\[
    \begin{pmatrix} E & F\\ F & G \end{pmatrix} = \begin{pmatrix} 
    1 + \sinh^2 \lambda & 0\\
    0 & \cosh^2 \lambda
    \end{pmatrix} = \begin{pmatrix} 
    \cosh^2 \lambda & 0\\
    0 & \cosh^2 \lambda
    \end{pmatrix}
\]
Para calcular la segunda forma fundamental necesitaremos la aplicación de Gauss
que en este caso será:
\begin{align*}
    N : X &\rightarrow X^2\\
    p &\mapsto N\left( p \right) = \frac{\varphi_{\lambda} \times
    \varphi_{\theta}}{\left\lVert \varphi_{\lambda} \times \varphi_{\theta} \right\rVert}
\end{align*}
Además:
\[
\varphi_{\lambda} \times \varphi_{\theta} = \begin{vmatrix} 
    i & j & k\\
    \sinh \lambda \cos \theta & \sinh \lambda \sin \theta & 1\\
    -\cosh \lambda \sin \theta & \cosh \lambda \cos \theta & 0
\end{vmatrix} = \left( -\cosh \lambda \cos \theta, -\cosh \lambda \sin \theta,
\sinh \lambda \cosh \lambda \right)
\]
y
\[
\left\lVert \varphi_{\lambda} \times \varphi_{\theta} \right\rVert =
\sqrt{\cosh^2 \lambda \left( 1 + \sinh^2 \lambda \right)} = \cosh^2 \lambda
\]
