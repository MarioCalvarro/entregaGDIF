\section{Apartado 2}
En primer lugar, recordamos que la parametrización que hemos elegido es:
\[
\varphi\left( \lambda, \theta \right) = \left( \cosh \lambda \cos \theta,
\cosh \lambda \sin \theta, \lambda \right).
\]
Siendo la definición de primera forma fundamental:
\begin{align*}
    \mathbf{I}_p: T_pX &\rightarrow \mathbb{R}\\
    \left( v, v' \right) &\mapsto \langle v, v' \rangle
\end{align*}
tiene como matriz respecto de la base $\langle \varphi_{\lambda}\left( p \right),
\varphi_{\theta}\left( p \right) \rangle$
\[
    \begin{pmatrix} E & F\\ F & G \end{pmatrix} = \begin{pmatrix} 
    \varphi_{\lambda} \cdot \varphi_{\lambda} & \varphi_{\lambda} \cdot \varphi_{\theta}\\
    \varphi_{\theta} \cdot \varphi_{\lambda} & \varphi_{\theta} \cdot
    \varphi_{\theta}
    \end{pmatrix}
\]
Por tanto, como:
\begin{gather*}
    \varphi_{\lambda} = \left( \sinh \lambda \cos \theta, \sinh \lambda \sin
    \theta, 1 \right)\\
    \varphi_{\theta} = \left( -\cosh \lambda \sin \theta, \cosh \lambda \cos
    \theta, 0 \right)
\end{gather*}
Tenemos que:
\begin{align*}
    \varphi_{\lambda} \cdot \varphi_{\lambda} &= \sinh^2 \lambda \cos^2 \theta +
    \sinh^2 \lambda \sin^2 \theta + 1 = 1 + \sinh^2 \lambda\\
    \varphi_{\theta} \cdot \varphi_{\theta} &= \cosh^2 \lambda \sin^2 \theta +
    \cosh^2 \lambda \cos^2 \theta = \cosh^2 \lambda\\
    \varphi_{\lambda} \cdot \varphi_{\theta} = \varphi_{\theta} \cdot
    \varphi_{\lambda} &= -\cosh \lambda \sinh \lambda \cos \theta \sin \theta +
    \cosh \lambda \sinh \lambda \cos \theta \sin \theta = 0
\end{align*}
Con lo que al final la matriz queda como:
\[
    \begin{pmatrix} E & F\\ F & G \end{pmatrix} = \begin{pmatrix} 
    1 + \sinh^2 \lambda & 0\\
    0 & \cosh^2 \lambda
    \end{pmatrix} = \begin{pmatrix} 
    \cosh^2 \lambda & 0\\
    0 & \cosh^2 \lambda
    \end{pmatrix}
\]
Para calcular la segunda forma fundamental necesitaremos la aplicación de Gauss
que será:
\begin{align*}
    N : X &\rightarrow X^2\\
    p &\mapsto N\left( p \right) = \frac{\varphi_{\lambda} \times
    \varphi_{\theta}}{\left\lVert \varphi_{\lambda} \times \varphi_{\theta} \right\rVert}
\end{align*}
Calculando
\[
\varphi_{\lambda} \times \varphi_{\theta} = \begin{vmatrix} 
    i & j & k\\
    \sinh \lambda \cos \theta & \sinh \lambda \sin \theta & 1\\
    -\cosh \lambda \sin \theta & \cosh \lambda \cos \theta & 0
\end{vmatrix} = \left( -\cosh \lambda \cos \theta, -\cosh \lambda \sin \theta,
\sinh \lambda \cosh \lambda \right)
\]
y
\[
\left\lVert \varphi_{\lambda} \times \varphi_{\theta} \right\rVert =
\sqrt{\cosh^2 \lambda \left( 1 + \sinh^2 \lambda \right)} = \cosh^2 \lambda
\]
En definitiva nos queda que:
\[
N\left( p \right) = \left( - \frac{\cos \theta}{\cosh \lambda}, -\frac{\sin
\theta}{\cosh \lambda}, \frac{\sinh \lambda}{\cosh \lambda} \right)
\]
Con este valor podemos calcular entonces la segunda forma fundamental que
recordemos definíamos como:
\begin{align*}
    \mathbf{II}_p : T_pX &\rightarrow \mathbb{R}\\
    \left( v, v' \right) &\mapsto dN_p\left( v \right) \cdot v'
\end{align*}
Con la siguiente matriz respecto de la base $\langle \varphi_{\lambda},
\varphi_{\theta} \rangle$:
\[
    \begin{pmatrix} e & f\\ f & g \end{pmatrix} = \begin{pmatrix} 
    N \cdot \varphi_{\lambda \lambda} & N \cdot \varphi_{\lambda \theta}\\
    N \cdot \varphi_{\theta \lambda} & N \cdot \varphi_{\theta \theta}
\end{pmatrix}
\]
Realizamos entonces los cálculos necesarios:
\begin{gather*}
    \varphi_{\lambda \lambda} = \left( \cosh \lambda \cos \theta, \cosh \lambda
    \sin \theta, 0 \right) \qquad N \cdot \varphi_{\lambda \lambda} = -1\\
    \varphi_{\theta \theta} = \left( - \cosh \lambda \cos \theta, -\cosh
    \lambda \sin \theta, 0 \right) \qquad N \cdot \varphi_{\theta \theta} = 1\\
    \varphi_{\lambda \theta} =\footnote{Lema de Schwarz} \varphi_{\theta
    \lambda} = \left( -\sinh \lambda \sin \theta, \sinh \lambda \cos \theta, 0
    \right) \qquad N \cdot \varphi_{\lambda \theta} = N \cdot \varphi_{\theta
    \theta} = 0\\
\end{gather*}
Por lo que al final nos queda la matriz:
\[
    \begin{pmatrix} e & f\\ f & g \end{pmatrix} = \begin{pmatrix} -1 & 0\\ 0 &
1 \end{pmatrix}
\]
Calculamos ahora la aplicación de Weingarten que se define como la derivada de
la aplicación de Gauss, es decir,
\[
d_p N: T_p X \rightarrow T_p X
\]
Haciendo uso de las formas fundamentales tenemos que:
\begin{gather*}
    \begin{pmatrix} e & f\\ f & g \end{pmatrix} = -dN_p \begin{pmatrix} E & F\\
F & G\end{pmatrix} \Leftrightarrow\\
    \begin{pmatrix} -1 & 0\\ 0 & 1 \end{pmatrix} = -\begin{pmatrix} 
    a_{11} & a_{21}\\ a_{12} & a_{22} \end{pmatrix}
    \begin{pmatrix} \cosh^2 \lambda & 0\\ 0 & \cosh^2 \lambda\end{pmatrix}
\end{gather*}
En definitiva,
\[
    dN_p = \begin{pmatrix} \frac{1}{\cosh^2 \lambda} & 0\\
    0 & -\frac{1}{\cosh^2 \lambda}\end{pmatrix}
\]
Por último, calculamos la curvatura de Gauss. Para ello, calcularemos los
autovalores de la aplicación de Weingarten para obtener $-k_1$ y $-k_2$:
\[
    \begin{vmatrix} \frac{1}{\cosh^2 \lambda} - \eta & 0\\
    0 & -\frac{1}{\cosh^2 \lambda} - \eta \end{vmatrix} = -\frac{1}{\cosh^4
    \lambda} - \eta = 0
\]
Por tanto,
\[
\eta = \pm \frac{1}{\cosh^2 \lambda} \Rightarrow \begin{cases}
    k_1 = \frac{1}{\cosh^2 \lambda} > 0\\
    k_2 = -\frac{1}{\cosh^2 \lambda} < 0
\end{cases} \Rightarrow \boxed{k = - \frac{1}{\cosh^4 \lambda} < 0}
\]
