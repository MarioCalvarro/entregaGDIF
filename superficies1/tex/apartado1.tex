\section{Apartado 1}
Recordamos que, para demostrar que $X$ es una superficie diferenciable podemos
simplemente ver que, considerando
\begin{align*}
    F: \mathbb{R}^3 &\rightarrow \mathbb{R}\\
    \left( x, y, z \right) &\mapsto x^2 + y^2 - \cosh^2z = 0
\end{align*}
y viendo que $X = \left\{ F\left( x, y, z \right) = 0 \right\}$, es suficiente
ver que $F \in C^{\infty)}$ y que $0$ es un valor regular de $F$, es decir, que
$\nabla F\left( p \right) \neq 0,\ \forall p \in X$. Que $F$ pertenezca a
$C^{\infty)}$ es trivial y, como $\nabla F = \left( 2x, 2y, 2\sinh z \cosh z
\right)$, tendremos que $\nabla F\left( x, y, z \right) = 0$ si, y sólo si,
\[
    \begin{cases}
        x = 0\\
        y = 0\\
        z = \sinh z \cosh z \Leftrightarrow \sinh z = 0 \text{ ó } \cosh z = 0
    \end{cases}
\]
Como 
\[
\sinh z = 0 \Leftrightarrow \frac{e^z - e^{-z}}{2} = 0 \Leftrightarrow e^z =
e^{-z} \Leftrightarrow z = 0
\]
y
\[
\cosh z = 0 \Leftrightarrow \frac{e^z + e^{-z}}{2} = 0 \Leftrightarrow e^z =
-e^{-z} \Leftrightarrow z = 0,
\]
tenemos que $\nabla F\left( x, y, z \right) = 0 \Leftrightarrow \left( x, y, z
\right) = 0$ que no pertenece a $X$ y, por lo tanto, $X$ es una superficie
diferenciable.

Veamos ahora una parametrización de la superficie. Si tomamos $u \in \mathbb{R}$
y $\theta \in \left[ 0, 2 \pi \right)$ y realizamos un cambio de variable, $x =
u \cos \theta,\ y = u \sin \theta$, tenemos que:
\[
u^2 - \cosh^2 z = 0
\]
y, por tanto,
\[
u^2 = \cosh^2 z.
\]
En definitiva, tenemos como parametrización: 
\begin{align*}
    \varphi: \left[ 0, 2 \pi \right) \times \mathbb{R} &\rightarrow X\\
    \left( \lambda, \theta \right) &\mapsto \left( \cosh \lambda \cos \theta, \cosh
    \lambda \sin \theta, \lambda \right)
\end{align*}
Sin embargo, esto no es del todo correcto puesto que nos interesa que el dominio
sea un abierto. Por esta razón utilizaremos dos parametrizaciones con dominio
abierto que cubran toda la superficie. Estarán definidas por las mismas
ecuaciones, pero una tendrá como dominio $\left( -\pi, \pi \right)$ y la otra
$\left( 0, 2 \pi \right)$. Con esto la primera cubrirá el hueco que deja la otra
en el $0$.

Ahora debemos comprobar que $\varphi$ es diferenciable (trivial por serlo sus
componentes), es un homeomorfismo con su imagen y que la aplicación diferencial
de $\varphi$ es inyectiva.

Veamos que es un homeomorfismo, es decir, que tanto $\varphi$ como
$\varphi^{-1}$ son continuas y biyectivas. La sobreyectividad se da al ser sobre
su imagen. Veamos la inyectividad. Si $\lambda_1 \neq \lambda_2 \in \mathbb{R}$
ó $\theta_1 \neq \theta_2 \in \left( 0, 2 \pi \right)$, supongamos que
$\varphi\left( \lambda_1, \theta_1 \right) = \varphi\left( \lambda_2, \theta_2
\right)$. Si $\lambda_1 \neq \lambda_2$ entonces la componente $z$ será distinta
directamente, por tanto, supongamos que $\lambda_1 = \lambda_2$ y $\theta_1 \neq
\theta_2$. Con esto tenemos que $\cosh \lambda_1 = \cosh \lambda_2$, es decir,
$\cosh \lambda \cos \theta_1 = \cosh \lambda \cos \theta_2
\Leftrightarrow\footnote{Podemos dividir entre $\cosh \lambda$ porque esta
función no se anula en ningún punto de su dominio.} \cos \theta_1 = \cos
\theta_2$, pero como estamos en un dominio de tamaño $< 2 \pi$, esto es
imposible y las componentes $x$ son distintas. De manera similar con la
componente $y$. Con esto $f$ es biyectiva.

Para ver ahora que $\varphi^{-1}$ es continua tomemos $q \in U$ y la proyección $\pi: \mathbb{R}^3 \rightarrow
\mathbb{R}^2$ tal que $\pi\left( x, y, z \right) = \left( x, y \right)$. Como
hemos visto que $\varphi$ es diferenciable, podemos aplicar el teorema de la
función inversa y obtenemos entornos $V_1 \subset U$ de $q$ y $V_2 \subset
\mathbb{R}^2$ de $\pi \circ \varphi\left( q \right)$ tal que se aplica de forma
difeomorfa $V_1$ a $V_2$ a través de $\pi \circ \varphi$. Como hemos visto que
$\varphi$ es biyectiva, al restringir a $\varphi\left( V_1 \right)$,
$\varphi^{-1} = \left( \pi \circ \varphi \right)^{-1} \circ \pi$ tenemos que
$\varphi^{-1}$ es composición de funciones continuas, es decir, es continua.

A continuación veremos que $d \varphi$ es inyectiva. Para ello veremos el rango
de la matriz de la aplicación lineal:
\[
    d \varphi = \begin{pmatrix}
    \sinh \lambda \cos \theta & -\cosh \lambda \sin \theta\\
    \sinh \lambda \sin \theta & \cosh \lambda \cos \theta\\
    1 & 0
    \end{pmatrix}
\]
es $2$ para todo los valores posibles. En primer lugar,
\[
\begin{vmatrix} 
    \sinh \lambda \cos \theta & -\cosh \lambda \sin \theta\\
    \sinh \lambda \sin \theta & \cosh \lambda \cos \theta
\end{vmatrix} = \sinh \lambda \cosh \lambda \left( \cos^2 \theta + \sin \theta
\right) = \sinh \lambda \cosh \lambda = 0
\]
Lo que solo se puede dar si $\lambda = 0$. Por tanto, solo nos queda comprobar
este caso:
\[
d \varphi \left( 0, \theta \right) = \begin{pmatrix} 
    0 & -\sin \theta\\
    0 & \cos \theta\\
    1 & 0
\end{pmatrix}
\]
con lo que el rango también es $2$ y la aplicación es inyectiva.

Por último, debemos comprobar que las dos parametrizaciones que hemos visto
antes cubren toda la superficie. Si llamamos $\varphi_1$ a la parametrización
con dominio $\left( 0, 2 \pi \right)$ y $\varphi_2$ a la que cubre $\left( -\pi,
\pi \right)$, tenemos que ver que $\varphi_1\left( U_1 \right) \cup
\varphi_2\left( U_2 \right) = X$, donde $U_1 = \left\{ \left( \lambda, \theta
    \right) : \lambda \in \mathbb{R},\ \theta \in \left( 0, 2 \pi \right)
    \right\}$ y $U_2 = \left\{ \left( \lambda, \theta \right) : \lambda \in
    \mathbb{R}, \theta \in \left( -\pi, \pi \right) \right\}$.
\begin{itemize}
\item[$\subset)$] Sea $\left( x, y, z \right) \in \varphi_1\left( U_1 \right)
    \cup \varphi_2\left( U_2 \right)$, entonces $\exists \left( \lambda, \theta
    \right) \in U_1 \cup U_2$ tal que 
    \[
        \varphi\left( \lambda, \theta \right) =
        \left( \cosh \lambda \cos \theta, \cosh \lambda \sin \theta, \lambda \right)
        = 
    \]
    debemos ver que se cumple que $x^2 + y^2 - \cosh^2 z = 0$ que sale de como
    hemos definido la parametrización. 

\item[$\supset)$] Sea $\left( x, y, z \right) \in X$. Si perteneciese a
    $\varphi_1\left( U_1 \right) \cup \varphi_2\left( U_2 \right)$ entonces debe
    existir $\left( \lambda, \theta \right) \in U_1 \cup U_2$ tal que
    $\varphi\left( \lambda, \theta \right) = \left( \cosh \lambda \cos \theta,
    \cosh \lambda \sin \theta, \lambda \right) = \left( x, y, z \right)$

    Si tenemos que $x \neq  0$, entonces:
    \[
    \begin{cases}
        \cosh \lambda \cos \theta = x\\
        \cosh \lambda \sin \theta = y\\
        \lambda = z
    \end{cases}
    \]
    lo que quiere decir que $\lambda = z$ y $\theta = \arctan\left( \frac{y}{x}
    \right)$, luego $\varphi\left( z, \arctan\left( \frac{y}{x} \right) \right)
    = \left( x, y, z \right)$.

    Si $x = 0$, $\left( 0, y, z \right) \in X \Rightarrow y = \pm \cosh z = \pm
    \cosh \lambda$, luego $\left( z, \frac{\pi}{2} \right), \left( z,
    \frac{3 \pi}{2} \right) \in U_0$ y su imagen por $\varphi$ es $\left( 0, y,
    z\right)$ porque: 
    \begin{gather*}
        \varphi\left( z, \frac{\pi}{2} \right) = \left( \cosh z \cos
        \frac{\pi}{2}, \cosh z \sin \frac{\pi}{2}, z \right) = \left( 0, \cosh
        z, z \right) = \left( 0, y, z \right)\\ 
        \varphi\left( z, \frac{3\pi}{2}
        \right) = \left( \cosh z \cos \frac{3\pi}{2}, \cosh z \sin \frac{3\pi}{2}, z
        \right) = \left( 0, -\cosh z, z \right) = \left( 0, y, z \right)\\
    \end{gather*}
\end{itemize}
En definitiva, cubrimos toda la superficie $X$.
