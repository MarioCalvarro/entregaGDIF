\section{Apartado 3}
Por el anterior apartado, hemos visto que $k < 0$, por lo tanto, los puntos de
la superficie son hiperbólicos. Por tanto, existe una base ortonormal $\langle
w_1, w_2 \rangle$ de $T_pX$ tal que $k_1 := k_n \left( w_1 \right) > 0, k_2 :=
k_n \left( w_2 \right) < 0$ y, al ser de distinto signo, tenemos que cada punto
es localmente un punto de silla. Como $k_n$ es continua, deberá existir un punto
en el que la curvatura cambie de signo:
\[
\exists w^* : k_n\left( w^* \right) = 0
\]
Veamos que cualquier recta que esté contenida en una superficie es una línea
asintótica. Esto quiere decir que, siendo $r$ una recta, $r$ (que es su propia
tangente) se encuentra es una dirección asintótica ($k_n = 0$).

Por tanto, sea $r: I \rightarrow X$ una curva parametrizada por la longitud de
arco (p.p.a) de la traza de una recta contenida en $X$. Al ser una recta,
sabemos que $r'' = 0$ y, por tanto, $k_{n_r} = k_r \cos \left(
\mathrm{ang}\left( \overrightarrow{n}, N \right) \right) = 0$ con lo que la
curvatura normal de todos los puntos de $r$ es nula y, en conclusión, $r$ es una
línea asintótica.
