\section{Apartado 3}
Por el anterior apartado, hemos visto que $k < 0$, por lo tanto, los puntos de
la superficie son hiperbólicos. Por tanto, existe una base ortonormal $\langle
w_1, w_2 \rangle$ de $T_pX$ tal que $k_1 := k_n \left( w_1 \right) > 0, k_2 :=
k_n \left( w_2 \right) < 0$ y, al ser de distinto signo, tenemos que cada punto
es localmente un punto de silla. Como $k_n$ es continua, deberá existir un punto
en el que la curvatura cambie de signo:
\[
\exists w^* : k_n\left( w^* \right) = 0.
\]
Veamos que cualquier recta que esté contenida en una superficie es una línea
asintótica. Esto quiere decir que, siendo $r$ una recta, $r$ (que es su propia
tangente) se encuentra en una dirección asintótica ($k_n = 0$). Además, esta
recta podrá existir en esta superficie porque, como acabamos de ver, existe un
$w^*$ con curvatura normal nula.

Por tanto, sea $r: I \rightarrow X$ una curva parametrizada por la longitud de
arco (p.p.a) con traza una recta contenida en $X$. Al ser una recta,
sabemos que $r'' = 0$ y, por tanto, $k_{n_r} = k_r \cos \left(
\mathrm{ang}\left( \overrightarrow{n}, N \right) \right) = 0$ con lo que la
curvatura normal de todos los puntos de $r$ es nula y, en conclusión, $r$ es una
línea asintótica. Veamos entonces las direcciones de las líneas asintóticas (que
pueden o, no ser, rectas, pero son las únicas posibles) que están contenidas en
$X$.

Sabiendo que $\mathbf{II}_p = 0$ en las direcciones asintóticas tenemos que:
\[
\mathrm{ker} \left( \mathbf{II}_p \right): 0 = -dN_p \left( x, y \right) \cdot
\left( x, y \right) = \left( \frac{x}{\cosh^2 \lambda}, -\frac{y}{\cosh^2
\lambda} \right) \cdot \left( x, y \right) = -\frac{1}{\cosh^2 \lambda} \left(
x^2 - y^2 \right).
\]
Es decir, $\pm x = y$ o, lo que es lo mismo, $\left( 1, 1 \right)$ ó $\left( 1,
-1\right)$. Utilizando estos puntos y la matriz de $d \varphi$ calculamos las
direcciones asintóticas: 
\[
\begin{cases}
    \begin{pmatrix} 
    \sinh \lambda \cos \theta & -\cosh \lambda \sin \theta\\
    \sinh \lambda \sin \theta & \cosh \lambda \cos \theta\\
    1 & 0
    \end{pmatrix} \begin{pmatrix} 1 \\ 1 \end{pmatrix} &= 
\underbrace{\left( \sinh \lambda \cos \theta - \cosh \lambda \sin \theta, \sinh \lambda
    \sin \theta + \cosh \lambda \cos \theta, 1
\right)}_{=: \overrightarrow{v_1}\left( \lambda, \theta \right)}\\

    \begin{pmatrix} 
    \sinh \lambda \cos \theta & -\cosh \lambda \sin \theta\\
    \sinh \lambda \sin \theta & \cosh \lambda \cos \theta\\
    1 & 0
    \end{pmatrix} \begin{pmatrix} 1 \\ -1 \end{pmatrix} &= 
    \underbrace{\left( \sinh \lambda \cos \theta + \cosh \lambda \sin \theta, \sinh \lambda
    \sin \theta - \cosh \lambda \cos \theta, 1 \right)}_{=:
    \overrightarrow{v_2}\left( \lambda, \theta \right)}
\end{cases}
\]
Analicemos el caso del punto $\left( 1, 0, 0 \right)$:
\[
\varphi\left( \lambda, \theta \right) = \left( 1, 0, 0 \right) \Rightarrow \begin{cases}
    \cosh \lambda \cos \theta = 1\\
    \cosh \lambda \sin \theta = 0\\
    \lambda = 0
\end{cases} \Rightarrow \begin{cases}
    \lambda = 0\\
    \theta = 0
\end{cases}.
\]
Para este punto las direcciones asintóticas son:
\begin{align*}
    \overrightarrow{v_1}\left( 0, 0 \right) &= \left( 0, 1, 1 \right) \\
    \overrightarrow{v_2}\left( 0, 0 \right) &= \left( 0, -1, 1 \right).
\end{align*}
Por lo que las líneas asintóticas serán:
\begin{align*}
    r_1 &:= \left\{ \left( 1, t, t \right) : t \in \mathbb{R} \right\}\\
    r_2 &:= \left\{ \left( 1, -t, t \right) : t \in \mathbb{R} \right\}.
\end{align*}
Veamos que no satisfacen la ecuación que define a $X$: ($x^2 + y^2 = \cosh^2 z$)
\begin{itemize}
    \item Primera línea, $r_1$, si se diese:
        \begin{align*}
            1 + t^2 = \cosh^2 t \Rightarrow t^2 = \cosh^2 t - 1 = \sinh^2 \Rightarrow t
            = \pm \sinh t.
        \end{align*}
        Por lo tanto,
        \begin{enumerate}
            \item Si $t = \sinh t$, lo cual no es cierto para, por ejemplo, $t =
                1$.
            \item Si $t = -\sinh t$, lo cual no es cierto para, por ejemplo, $t
                = 1$.
        \end{enumerate}
    \item Segunda línea, $r_2$, si se diese:
    \[
    1 + t^2 = \cosh^2 t
    \]
    y procedemos como en la anterior línea.
\end{itemize}
Como estas son las únicas posibles rectas que pasan por $\left( 1, 0, 0 \right)$
contenidas en $X$, pero realmente no son rectas, no existe ninguna.
