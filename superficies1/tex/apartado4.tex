\section{Apartado 4}
Sea $\alpha\left( t \right) = \varphi\left( \lambda \left( t \right), \theta
\left( t \right) \right)$ una línea asintótica. Derivando tenemos que:
\[
\alpha'\left( t \right) = \lambda'\left( t \right) \varphi_{\lambda}\left(
\lambda\left( t \right), \theta\left( t \right) \right) + \theta'\left( t
\right) \varphi_{\theta} \left( \lambda\left( t \right), \theta\left( t \right)
\right).
\]
Por lo que las coordenadas de $\alpha'$ en la base $\langle \varphi_{\lambda}
\left( \lambda\left( t \right), \theta\left( t \right) \right), \varphi_{\theta}
\left( \lambda \left( t \right), \theta\left( t \right) \right)\rangle$ serán
$\left( \lambda'\left( t \right), \theta'\left( t \right) \right)$ y, al ser
asintótico, $\mathbf{II}_p\left( \lambda'\left( t \right), \theta'\left( t
\right) \right) = 0$, lo que equivale a que:
\[
\begin{pmatrix}  \lambda'\left( t \right) & \theta'\left( t \right)  \end{pmatrix} \begin{pmatrix}
-1 & 0\\ 0 & 1 \end{pmatrix} \begin{pmatrix} \lambda'\left( t \right) \\
\theta'\left( t \right) \end{pmatrix} = 0 \Rightarrow -\lambda'\left( t
\right)^2 + \theta'\left( t \right)^2 = 0;\ \forall t \in \mathbb{R}.
\]
Es decir, que para todo $t \in \mathbb{R}$,
\[
\begin{cases}
    \lambda'\left( t \right) = \theta'\left( t \right) \Rightarrow \lambda\left(
    t\right) = \theta\left( t \right) + C_1\\
    \lambda'\left( t \right) = -\theta'\left( t \right) \Rightarrow
    \lambda\left( t\right) = -\theta\left( t \right) + C_2\\
\end{cases}.
\]
En el caso concreto del punto $\left( 1, 0, 0 \right)$, tenemos que
$\varphi^{-1}\left( 1, 0, 0 \right) = \left( 0, 0 \right)$. Suponiendo que
$\alpha\left( 0 \right) = \left( 1, 0, 0 \right)$, por el anterior ejercicio,
sabemos que $\lambda\left( 0 \right) = 0 = \theta\left( 0 \right)$ por lo que
$C_1 = C_2 = 0$ y, entonces, $\lambda\left( t \right) = \theta\left( t \right)$
ó $\lambda\left( t \right) = -\theta\left( t \right)$. En definitiva:
\begin{enumerate}
    \item Si $\lambda\left( t \right) = \theta\left( t \right)$, tenemos que la
        parametrización es:
    \[
        \alpha\left( t \right) = \varphi\left( \lambda\left( t \right),
        \lambda\left( t \right) \right) = \boxed{\left( \cosh\left(
            \lambda\left( t \right) \right)\cos \left( \lambda\left( t \right)
        \right), \cosh\left( \lambda\left( t \right) \right)\sin \left(
        \lambda\left( t \right) \right), \lambda\left( t \right) \right)}.
    \]
    \item Si $\lambda\left( t \right) = -\theta\left( t \right)$, tenemos que la
        parametrización es:
    \[
        \alpha\left( t \right) = \varphi\left( \lambda\left( t \right),
        -\lambda\left( t \right) \right) = \boxed{\left( \cosh\left(
        \lambda\left( t \right) \right)\cos \left( -\lambda\left( t \right)
        \right), \cosh\left( \lambda\left( t \right) \right)\sin \left(
        -\lambda\left( t \right) \right), \lambda\left( t \right) \right)}.
    \]
\end{enumerate}
Que son líneas asintóticas no rectas por el anterior ejercicio.
