\section{Estudio de posibles geodésicas}
En primer lugar, veamos que las curvas dadas son las semicircunferencias
pedidas. Esto es equivalente a que, para cualquier $a$ y $r$ dados, existe un
punto en $v = 0$ que está a la misma distancia de todos los puntos de la curva.
Además, la curva debe ser exactamente una semicircunferencia. Antes de empezar
a hacer cálculos daremos otra forma de expresar las componentes hiperbólicas de
la curva:
\[
\tanh s = \frac{\sinh s}{\cosh s} = \frac{e^s - e^{-s}}{e^{s} + e^{-s}}\qquad
\sech s = \cosh^{-1} s = \frac{2}{e^s + e^{-s}}.
\]
Con esto empezaremos calculando los límites de la curva cuando $s$ tiende a
infinito:
\begin{align*}
    \lim_{s \rightarrow \infty} \left( u\left( s \right), v\left( s \right)
    \right) &= \lim_{s \rightarrow \infty} \left( a + r \cdot \frac{e^s -
    e^{-s}}{e^s + e^{-s}}, r \cdot \frac{2}{e^s + e^{-s}} \right)\\
    &= \lim_{s \rightarrow \infty} \left( a + r \cdot \frac{e^s}{e^s}, r
    \cdot \frac{2}{e^s} \right) = \left( a + r, 0 \right).\\
    \lim_{s \rightarrow -\infty} \left( u\left( s \right), v\left( s \right)
    \right) &= \lim_{s \rightarrow -\infty} \left( a + r \cdot \frac{e^s -
    e^{-s}}{e^s + e^{-s}}, r \cdot \frac{2}{e^s + e^{-s}} \right)\\
    &= \lim_{s \rightarrow -\infty} \left( a - r \cdot \frac{e^{-s}}{e^{-s}}, r
    \cdot \frac{2}{e^s} \right) = \left( a - r, 0 \right).\\
\end{align*}
A su vez, $v\left( s \right) = r \cdot \frac{2}{e^s + e^{-s}} \begin{cases} > 0,
&\text{si } r > 0\\ < 0, &\text{si } r < 0 \end{cases}$. Con esto hemos visto
que la curva tiende (en $-\infty$ y $+\infty$) a dos puntos con $v = 0$  sin
cruzar antes ningún otro con $v = 0$. Es decir, que si encontramos un punto en
$v = 0$ que equidiste a todos los de la curva, aseguraremos que se trata de una
semicircunferencia centrada en $v = 0$.

Para obtener un posible candidato a centro de la supuesta semicircunferencia
tenemos un par de datos, tiene que tener $v = 0$ y equidistar de los dos límites
que acabamos de calcular. Con esto un posible centro sería el punto $\left( a, 0
\right)$. Comprobemos que efectivamente lo es. Para ello vamos a ver que el
cuadrado (y, por tanto, sin el cuadrado también) de la distancia euclídea del
posible centro a cualquier punto de la curva es constante:
\[
\left\lvert \left( x - a , y \right) \right\rvert^2 \stackrel{?}{\equiv}
\mathrm{const.}\ \forall \left( x, y \right) \in \left( u\left( s \right), v\left( s
\right) \right).
\]
Sustituyendo el valor de un punto de la curva, tenemos que:
\begin{align*}
    \left\lvert \left( a + r \tanh s - a, r \sech s \right) \right\rvert^2 &=
    r^2 \left( \tanh^2 s + \sech^2 s \right)\\ 
    &= r^2 \cdot \frac{\sinh^2 s + 1}{\cosh^2 h}\\
    &= r^2 \cdot \frac{\cosh^2 s - 1 + 1}{\cosh^2 s} = r^2 \equiv \mathrm{const.}
\end{align*}
Y con esto hemos probado que estas curvas son semicircunferencias con centro
en $v = 0$.

A continuación comprobaremos que las imágenes de estas curvas por $\varphi$
están parametrizadas por su longitud de arco. La imagen de la curva por
$\varphi$ será:
\[
\alpha\left( t \right) = \varphi\left( u\left( t \right), v \left( t \right) \right).
\]
Con lo que $\alpha'\left( t \right) = \varphi_u u'\left( t \right) + \varphi_v
v'\left( t \right)$ y $\left( u'\left( t \right), v'\left( t \right) \right)$
son las coordenadas de $\alpha'\left( t \right)$ sobre $\left\{ \varphi_u,
\varphi_v \right\}$. Por la definición de la \textit{Primera Forma Fundamental}
sabemos que:
\[
\left\lVert \alpha'\left( t \right) \right\rVert^2 = \mathrm{I}
\left( u'\left( t \right), v'\left( t \right) \right).
\]
Con lo que:
\[
L_{t_0}^t \left( \alpha \right) = \int_{t_0}^{t} \left\lVert \alpha'\left( s
\right) \right\rVert \mathrm{d}s = \int_{t_0}^{t}
\sqrt{\mathrm{I} \left( u'\left( s \right), v'\left( s
\right) \right)} \mathrm{d}s.
\]
Calculemos, pues, la primera forma fundamental para $\left( u'\left( t \right),
v'\left( t \right) \right)$:
\begin{align*}
\mathrm{I} \left( u'\left( t \right), v'\left( t
\right) \right) &= \left( u'\left( t \right), v'\left( t \right) \right)
\begin{pmatrix} \frac{1}{v^2} & 0\\ 0 & \frac{1}{v^2} \end{pmatrix}
\begin{pmatrix} u'\left( t \right)\\ v'\left( t \right) \end{pmatrix}\\ 
&= \frac{1}{v^2} \cdot \left( u'\left( t \right), v'\left( t \right) \right) \begin{pmatrix} u'\left( t \right)\\ v'\left( t \right) \end{pmatrix}\\
&= \frac{1}{v^2} \left( \left( u'\left( t \right) \right)^2 + \left( v'\left( t
\right) \right)^2 \right).
\end{align*}
Como $u'\left( t \right) = r \sech^2 t$ y $v'\left( t \right) = -r \sech t
\cdot \tanh t$, si sustituimos tenemos que:
\begin{align*}
\mathrm{I}\left( u'\left( t \right), v'\left( t
\right) \right) &= \frac{r^2}{r^2 \cdot \sech^2 t} \cdot \sech^2 t \left( \sech^2 t + \tanh^2 t
\right)\\
&= \sech^2 t + \tanh^2 t.
\end{align*}
En definitiva,
\begin{align*}
L_{t_0}^t \left( \varphi \right) &= \int_{t_0}^{t} \sqrt{\sech^2 t + \tanh^2
t} \mathrm{d}t\\
&= \int_{t_0}^{t} \sqrt{\frac{1}{\cosh^2 t} + \frac{\sinh^2 t}{\cosh^2 t}}
\mathrm{d}t\\
&= \int_{t_0}^{t} \sqrt{\frac{1 + \sinh^2 t}{\cosh^2 t}} \mathrm{d}t\\
&= \int_{t_0}^{t}  \mathrm{d}t= t - t_0.
\end{align*}
Con lo que queda probado que esta parametrizada por longitud de arco.

Por último, comprobemos que son geodésicas. Como la curva está parametrizada por
longitud de arco, será geodésica si cumple que:
\[
\begin{cases}
    u''\left( t \right) &= - \Gamma_{11}^1 u'\left( t \right)^2 - 2 \Gamma_{12}^1
    u'\left( t \right) v'\left( t \right) - \Gamma_{22}^1 v'\left( t \right)^2\\
    v''\left( t \right) &= - \Gamma_{11}^2 u'\left( t \right)^2 - 2 \Gamma_{12}^2
    u'\left( t \right) v'\left( t \right) - \Gamma_{22}^2 v'\left( t \right)^2\\
\end{cases}.
\]
Por tanto, para resolver este sistema de ecuaciones necesitamos saber el valor
de los símbolos de Christoffel. Para ello usamos su relación con la
matriz de la \textit{Primera Forma Fundamental}:
\begin{align*}
    \begin{cases}
        \frac{1}{2} E_u &= E \Gamma_{11}^1 + F \Gamma_{11}^2\\
        F_u - \frac{1}{2} E_v &= F \Gamma_{11}^1 + G \Gamma_{11}^2
    \end{cases} &\Rightarrow \begin{cases}
        0 &= \frac{1}{v^2} \Gamma_{11}^1\\
        \frac{1}{v^3} &= \frac{1}{v^2} \Gamma_{11}^2
        \end{cases} \xRightarrow{v > 0} \begin{cases}
        \Gamma_{11}^1 &= 0\\
        \Gamma_{11}^2 &= \frac{1}{v}
    \end{cases}\\
    \begin{cases}
        \frac{1}{2}E_v &= E \Gamma_{12}^1 + F \Gamma_{12}^2\\
        \frac{1}{2} G_u &= F \Gamma_{12}^1 + G \Gamma_{12}^2
    \end{cases} &\Rightarrow \begin{cases}
        -\frac{1}{v^3} &= \frac{1}{v^2} \Gamma_{12}^1\\
        0 &= \frac{1}{v^2} \Gamma_{12}^2
    \end{cases} \xRightarrow{v > 0} \begin{cases}
        \Gamma_{12}^1 &= -\frac{1}{v}\\
        \Gamma_{12}^2 &= 0
    \end{cases}\\
    \begin{cases}
        F_v - \frac{1}{2} G_u &= E \Gamma_{22}^1 + F \Gamma_{22}^2\\
        \frac{1}{2} G_v &= F \Gamma_{22}^1 + G \Gamma_{22}^2
    \end{cases} &\Rightarrow \begin{cases}
        0 &= \frac{1}{v^2} \Gamma_{22}^1\\
        -\frac{1}{v^3} &= \frac{1}{v^2} \Gamma_{22}^2
        \end{cases} \xRightarrow{v > 0} \begin{cases}
        \Gamma_{22}^1 &= 0\\
        \Gamma_{22}^2 &= -\frac{1}{v}
    \end{cases}.
\end{align*}
Con lo que solo queda comprobar si se verifican las ecuaciones de las geodésicas
para esta curva. Calculemos las derivadas de $u$ y $v$:
\begin{align*}
    u'\left( t \right) &= r \sech^2 t & u''\left( t \right) &= 2 r
    \sech t \cdot \left( - \sech t \cdot \tanh t \right) = -2 r \sech^2 t \cdot
    \tanh t\\
    v'\left( t \right) &= -r \sech t \cdot \tanh t & v''\left( t \right) &= r
    \sech t \cdot \tanh^2 t - r \sech^3 t = r\sech t \left( \tanh^2 t - \sech^2
    t \right).
\end{align*}
Y ahora sustituimos en las ecuaciones de las geodésicas con los valores que
hemos obtenido para los símbolos de Christoffel:
\[
\begin{cases}
    u''\left( t \right) &= -\frac{2}{v}r^2 \sech^3 t \cdot \tanh t\\
    v''\left( t \right) &= -\frac{1}{v} r^2 \sech^4 t + \frac{1}{v} r^2 \sech^2 t
    \cdot \tanh^2 t
\end{cases} \xRightarrow{v = r \sech t} \begin{cases}
    u''\left( t \right) = -2 r \sech^2 \left( t \right) \cdot \tanh t\\
    v''\left( t \right) = r \sech t \left( \tan^2 t - \sech^2 t \right)
\end{cases}.
\]
Y con esto hemos comprobado que es una geodésica.
