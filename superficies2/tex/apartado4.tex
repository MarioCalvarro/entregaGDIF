\section{Estudio de otras posibles geodésicas}
Supongamos que existen otras curvas geodésicas aparte de las dos posibles que
hemos visto. Esto significa que existen dos puntos distintos entre sí tales que
la curva más corta que los une es distinta a las semirrectas verticales o  a las
semicircunferencias con centro en $v = 0$.

Claramente, por la unicidad de las geodésicas, si los dos puntos tienen la misma
coordenada $x$, necesariamente su geodésica es una de las ya vistas. Por tanto,
podemos suponer que tienen distinta coordenada $x$. En definitiva, si vemos que
dos puntos cualesquiera con distinta $x$ se encuentran en una semicircunferencia
con centro en $v = 0$, tendremos el resultado que buscamos.
