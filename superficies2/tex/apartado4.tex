\section{Estudio de otras posibles geodésicas}
Supongamos que existen otras curvas geodésicas aparte de las dos posibles que
hemos visto. Esto significa que existen dos puntos distintos entre sí tales que
la curva más corta que los une es distinta a las semirrectas verticales o  a las
semicircunferencias con centro en $v = 0$.

Claramente, por la unicidad de las geodésicas, si los dos puntos tienen la misma
coordenada $x$, necesariamente su geodésica es una de las ya vistas. Por tanto,
podemos suponer que tienen distinta coordenada $x$. Sin embargo, como entre
cualesquiera dos puntos del semiplano superior de $\mathbb{R}^2$, con distinta
$x$, pasa una sola semicircunferencia con centro en el eje $y = 0$ y las
geodésicas son únicas, no es posible que haya otras geodésicas distintas a las
que hemos visto.
