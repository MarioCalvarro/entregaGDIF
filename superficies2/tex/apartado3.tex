\section{Estudio de las rectas verticales}
Empecemos dando una parametrización para las rectas verticales (es decir,
aquellas con su coordenada $x$ constante):
\[
\gamma\left( t \right) = \left( u\left( t \right), v\left( t \right) \right) =
\left( a, re^t \right)
\]
Por los anteriores cálculos sabemos que $\mathrm{I}\left( u'\left( t \right), v'\left( t \right) \right) =
\frac{1}{v^2}\left( \left( u'\left( t \right) \right)^2 + \left( v'\left( t
\right) \right)^2 \right)$. Como en este caso $u'\left( t \right) = 0$ y
$v'\left( t \right) = r e^t$, tenemos que:
\[
\mathrm{I} \left( u'\left( t \right), v'\left( t
\right) \right) = 1.
\]
Y, de esta forma,
\[
L_{t_0}^t \left( \gamma \right) = \int_{t_0}^{t}
\sqrt{\mathrm{I} \left( u'\left( t \right), v'\left(
t \right) \right)} \mathrm{d}t = \int_{t_0}^{t} 1 \cdot \mathrm{d}t = t - t_0.
\]
Con lo que, con esta parametrización, las rectas verticales ya están
parametrizadas por longitud de arco.

De nuevo, al estar parametrizadas por longitud de arco, podemos comprobar
directamente si son geodésicas viendo si cumplen las ecuaciones que hemos
planteado para el anterior caso. Además, como los símbolos de Christoffel solo
dependen de la Primera Forma Fundamental, no será necesario volverlos a
calcular. En definitiva, calculemos en primer lugar las derivadas de $u$ y $v$:
\begin{align*}
    u'\left( t \right) &= 0 & u''\left( t \right) &= 0\\
    v'\left( t \right) &= r e^t & v''\left( t \right) &= re^t,
\end{align*}
y sustituyamos en las ecuaciones de las geodésicas:
\[
\begin{cases}
    u''\left( t \right) &= \frac{2}{v} \cdot 0\\
    v''\left( t \right) &= - \frac{1}{v} \cdot 0^2 + \frac{1}{v} r^2 e^{2t} 
\end{cases} \xRightarrow{v = re^t} \begin{cases}
    u''\left( t \right) = 0\\
    v''\left( t \right) = re^t
\end{cases}.
\]
Con lo que es una geodésica.
