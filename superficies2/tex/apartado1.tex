\section{Curvatura de Gauss}
Veamos que, efectivamente, la curvatura de Gauss de esta superficie es
constantemente $-1$. Por hipótesis, sabemos que la matriz de la \textit{Primera
Forma Fundamental} de esta superficie es:
\[
    \begin{pmatrix} E & F\\ F & G \end{pmatrix} = \begin{pmatrix} \frac{1}{v^2} & 0\\
    0 & \frac{1}{v^2}\end{pmatrix}.
\]
Además, gracias a que $F = 0$, podemos aplicar el ejercicio $7$ de la hoja $6$
que nos indica una forma de la curvatura de Gauss:
\[
K = -\frac{1}{2 \sqrt{EG}}\left( \left( \frac{E_v}{\sqrt{EG}} \right)_v + \left(
\frac{G_u}{\sqrt{EG}}\right)_u \right).
\]
Por lo que solo queda calcular los distintos términos:
\begin{align*}
    \sqrt{EG} &= \sqrt{\frac{1}{v^4}} = \frac{1}{v^2} & E_v &= -\frac{2}{v^3} &
    G_u &= 0\\
        & & \left( \frac{E_v}{\sqrt{EG}} \right)_v &= \left( -\frac{2}{v}
        \right)_v = \frac{2}{v^2}.
\end{align*}
Sustituyendo,
\[
K = -\frac{v^2}{2} \left( \frac{2}{v^2} + 0 \right) = - \frac{v^2}{2} \cdot
\frac{2}{v^2} = -1.
\]
Que es lo que buscábamos.
