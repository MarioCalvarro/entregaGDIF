\section{Enunciado}
Demuestra que una superficie parametrizada tal que $E = \frac{1}{v^2}$, $F = 0$,
$G = \frac{1}{v^2}$ para todo $\left( u, v \right) \in \mathbb{R}^2$ con $v > 0$
tiene una curvatura de Gauss constante igual a $-1$. Comprueba que las curvas de
la forma
\[
\left( u\left( s \right), v\left( s \right) \right) = \left( a + r \tanh \left(
s\right), r \sech\left( s \right) \right)
\]
son las semicircunferencias con centro en $v = 0$ y que su imagen por $\varphi$
están parametrizadas por su longitud de arco y son geodésicas. Parametriza por
arco las semirrectas verticales y comprueba que también son geodésicas. ¿Puede
haber más geodésicas?
